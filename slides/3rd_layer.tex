  \subsection{Network}
  \begin{frame}
    \frametitle{Aims}
      \begin{itemize}
        \item Interface transport layer,\pause
	\item Host addressing,\pause
        \item End-to-end packet transmission (data link? Connectionless? Switch? Router?),\pause
        \item Routing, load balancing
      \end{itemize}
  \end{frame}
  \subsubsection{IP addressing}
  \begin{frame}
    \frametitle{Concepts}
      \begin{itemize}
        \item IP addressing fundamentals,\pause
        \item Classfull IP addressing,\pause
        \item Subnet masks,\pause
        \item Variable length subnet masks (VLSM),\pause
        \item Classless inter-domain routing (CIDR).
      \end{itemize}
  \end{frame}

  \begin{frame}
    \frametitle{IP addressing fundamentals}
    \begin{block}{IP address}
      \begin{figure}
        \centering
        \begin{tabular}{|c|c|}
          \multicolumn{2}{c}{32 bits (4x4 bytes)} \\ \hline
           \multicolumn{2}{|c|}{\color{brown}mask} \\ \hline
          \color{brown}Networks part & \color{blue}Host part \\ \hline
        \end{tabular}
        \caption{IP address parts}
        \label{fig:inside_ip_address}
      \end{figure}
    \end{block}
  \end{frame}

  \begin{frame}
    \frametitle{IP addressing fundamentals}
    \begin{block}{Masks}
      \begin{itemize}
        \item Separates network and host bits,\pause
        \item MSB \textbf{always} are ones and then zeros! 255.254.255.0 is not possible,\pause
        \item Indicates how many bits are used for the network part:
        \begin{itemize}
          \item A 8-bit mask leaves 24 bits for the hosts,
          \item A 16-bit mask leaves 16 bits for the hosts,
          \item A 24-bit mask leaves 8 bits for the hosts,
          \item A N-bit mask leaves 32-N bits for the hosts.
        \end{itemize}\pause
        \item Two different mask (differences seen further):
        \begin{itemize}
          \item Network mask,
          \item Subnet mask.
        \end{itemize}
      \end{itemize}
    \end{block}
  \end{frame}
  \begin{frame}
    \frametitle{IP addressing fundamentals}
    \begin{block}{IP address}
      \begin{figure}
        \centering
        \begin{tabular}{|c|c|}
          \multicolumn{2}{c}{32 bits (4x4 bytes)} \\ \hline
          \uncover<2->{\color{brown}ones mask} & \uncover<2->{\color{blue}zeros mask} \\ \hline
          \color{brown}Networks part & \color{blue}Host part \\ \hline
        \end{tabular}
        \caption{IP address parts and mask}
        \label{fig:inside_ip_address_mask}
      \end{figure}
    \end{block}
  \end{frame}

  \begin{frame}
    \frametitle{IP addressing fundamentals}
    \begin{block}{Is that a host?}
      \begin{itemize}
        \item Network address,\pause
        \item Nodes,\pause
        \item Broadcast address.\pause
      \end{itemize}
    \end{block}
    \begin{block}{Within the same network}
      \begin{itemize}
        \item All addresses have the same network bits,\pause
        \item All nodes have different host bits,\pause
        \item Network address has zeros for host bits,\pause
        \item Broadcast address has ones for host bits.
      \end{itemize}
    \end{block}
  \end{frame}

  \begin{frame}
    \frametitle{Example: network 1}
    \begin{figure}
        \centering
      \begin{tabular}{|r|cccc|}
        \hline
        \multirow{2}{*}{Mask /24} & {\color{brown}255} & {\color{brown}255} & {\color{brown}255} & {\color{brown}0} \\
        & {\color{brown}11111111} & {\color{brown}11111111} & {\color{brown}11111111} & {\color{brown}00000000} \\ \hline
        \multirow{2}{*}{Network address} & \color{brown}192 & \color{brown}168 & \color{brown}1 & \color{blue}0 \\
        & \color{brown}11000000 & \color{brown}10101000 & \color{brown}00000001 & \color{blue}00000000 \\ \hline
        \multirow{2}{*}{First nodes address} & \color{brown}192 & \color{brown}168 & \color{brown}1 & \color{blue}1 \\
        & \color{brown}11000000 & \color{brown}10101000 & \color{brown}00000001 & \color{blue}00000001 \\ \hline
        \multirow{2}{*}{Last nodes address} & \color{brown}192 & \color{brown}168 & \color{brown}1 & \color{blue}254 \\
        & \color{brown}11000000 & \color{brown}10101000 & \color{brown}00000001 & \color{blue}11111110 \\ \hline
        \multirow{2}{*}{Broadcast address} & \color{brown}192 & \color{brown}168 & \color{brown}1 & \color{blue}255 \\
        & \color{brown}11000000 & \color{brown}10101000 & \color{brown}00000001 & \color{blue}11111111 \\ \hline
      \end{tabular}
      \caption{IP address example 1}
    \end{figure}
  \end{frame}

  \begin{frame}
    \frametitle{Example: network 2}
    \begin{figure}
        \centering
      \begin{tabular}{|r|cccc|}
        \hline
        \multirow{2}{*}{Mask /16} & {\color{brown}255} & {\color{brown}255} & {\color{brown}0} & {\color{brown}0} \\
        & {\color{brown}11111111} & {\color{brown}11111111} & {\color{brown}00000000} & {\color{brown}00000000} \\ \hline
        \multirow{2}{*}{Network address} & \color{brown}172 & \color{brown}17 & \color{blue}0 & \color{blue}0 \\
        & \color{brown}10101100 & \color{brown}00010001 & \color{blue}00000000 & \color{blue}00000000 \\ \hline
        \multirow{2}{*}{First nodes address} & \color{brown}172 & \color{brown}17 & \color{blue}0 & \color{blue}1 \\
        & \color{brown}10101100 & \color{brown}00010001 & \color{blue}00000000 & \color{blue}00000001 \\ \hline
        \multirow{2}{*}{Last nodes address} & \color{brown}172 & \color{brown}17 & \color{blue}255 & \color{blue}254 \\
        & \color{brown}10101100 & \color{brown}00010001 & \color{blue}11111111 & \color{blue}11111110 \\ \hline
        \multirow{2}{*}{Broadcast address} & \color{brown}172 & \color{brown}17 & \color{blue}255 & \color{blue}255 \\
        & \color{brown}10101100 & \color{brown}00010001 & \color{blue}11111111 & \color{blue}11111111 \\ \hline
      \end{tabular}
      \caption{IP address example 2}
    \end{figure}
  \end{frame}

  \begin{frame}
    \frametitle{Formula}
    \begin{block}{How many \sout{hosts} nodes with a N-bit mask?}
      $2^{32-N}-2$\pause, the $-2$ moves out network and broadcast addresses which are not nodes.\pause
      \begin{itemize}
        \item 24-bit mask: $2^{32-24}-2 = 2^{8}-2 = 254$ nodes \pause
        \item 16-bit mask: $2^{32-16}-2 = 2^{16}-2 = 65.534$ nodes \pause
        \item 8-bit mask: $2^{32-8}-2 = 2^{24}-2 = 16.777.214$ nodes
      \end{itemize}
    \end{block}
  \end{frame}

  \begin{frame}
    \frametitle{Public and private addresses}
    \begin{block}{Public}
      \begin{itemize}
        \item Most of IP addresses \pause
        \item Registered ISP and large organizations inherit blocks of public addresses from IANA\footnote{Internet Assigned Numbers Authority} \pause
        \item Usage of not registered public addresses is forbidden.
      \end{itemize}
    \end{block}
    \begin{block}{Private}
      \begin{itemize}
        \item Privates addresses are A, B and C classes (see after)\pause
        \item No registration needed \pause
        \item Not routed across the Internet \pause
        \item Proxy, NAT and private addresses solved IPv4 shortage.
      \end{itemize}
    \end{block}
  \end{frame}

  
  \begin{frame}
    \frametitle{Classful IP Addressing}
    \begin{figure}
      \centering
      \begin{tabular}{|r||c|c|c|}
        \hline
        Class & A & B & C \\ \hline \hline
        First octet & 1 - 126 & 128 - 191 & 192 - 223 \\ \hline
        First octet pattern 0b& 0* & 10* & 110* \\ \hline
        \multirow{2}{*}{\color{brown}Network mask} & 255.0.0.0 & 255.255.0.0 & 255.255.255.0\\
         & /8 & /16 & /24 \\ \hline
        \multirow{2}{*}{IP addresses range} & 1.0.0.0 & 128.0.0.0 & 192.0.0.0\\
        & 126.0.0.0 & 191.255.0.0 & 223.255.255.0 \\ \hline
        Number of nodes & 16777214 & 65534 & 254 \\ \hline
      \end{tabular}
      \caption{Three main classes}
    \end{figure}
    Where did 127.0.0.0/8 go ?!
  \end{frame}

  \begin{frame}
    \frametitle{Classful IP Addressing}
    \begin{block}{Class D}
      \begin{itemize}
	\item First octet: 224 - 239 \pause
	\item First octet pattern: 1110* \pause
	\item Theses IP addresses are multicast addresses.\pause
      \end{itemize}
    \end{block}
    \begin{block}{Class E}
      \begin{itemize}
	\item Everything left \pause
	\item Experimental class.
      \end{itemize}
    \end{block}
  \end{frame}
  \begin{frame}
    \begin{block}{Reserved addresses}
      \begin{itemize}
	\item 0.0.0.0 used in routing (seen further) \pause
	\item 127.0.0.0/8: loopback addresses (127.0.0.1 - 127.255.255.254).
      \end{itemize}
    \end{block}
  \end{frame}
