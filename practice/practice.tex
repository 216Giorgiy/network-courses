\documentclass[11pt]{article}
\usepackage[utf8x]{inputenc}
\usepackage[english]{babel}
\usepackage{graphicx}
\usepackage{wrapfig}
\usepackage[margin=3cm, tmargin=2cm]{geometry}
\usepackage{color}
\usepackage{hyperref}

\begin{document}
 \bibliographystyle{plain}%Choose a bibliograhpic style
\title{Network training}
\date{Fall 2014}
\author{Maël Auzias}
\maketitle

\tableofcontents
\pagebreak


\section{HTTP example}
\subsection{Who are you? Where are you?}
What is your own IP address? What is your own MAC address? What do theses commands display?
\begin{verbatim}
#ifconfig
$curl ifconfig.me
$netstat -at
\end{verbatim}

\subsubsection{ARP}
Before we can access the Internet we need to know who/what is the gateway. What is the IP address of yours? What is the MAC address? What do theses commands display?
\begin{verbatim}
#route -n
#arp -a
\end{verbatim}

\section{Chat}
\verb"netcat" is a "network swiss army knife". By checking its man page how can you use it as a chat server/client (two nodes only).
\subsubsection{TCP}
Use the mode TCP of \verb"netcat" and try it. Can \verb"netstat" could, somehow, be handful for anything while waiting for connection?
\subsubsection{UDP}
Use the mode UDP of \verb"netcat" and try it. Explain a situation within the server could not receive every packet.

\end{document}
