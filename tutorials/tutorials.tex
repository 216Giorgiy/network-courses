\documentclass[11pt]{article}
\usepackage[utf8x]{inputenc}
\usepackage[english]{babel}
\usepackage{graphicx}
\usepackage{wrapfig}
\usepackage[margin=3cm, tmargin=2cm]{geometry}
\usepackage{color}
\usepackage{hyperref}

\begin{document}
 \bibliographystyle{plain}%Choose a bibliograhpic style
\title{Network training}
\date{Fall 2014}
\author{Maël Auzias}
\maketitle

\tableofcontents
\pagebreak


\section{Introduction}
\subsection{Classification}
Give a concrete example of each of the following kinds of networks (name some devices):
  \begin{enumerate}
    \item BAN,
    \item PAN,
    \item LAN,
    \item WAN.
  \end{enumerate}

\subsection{Topologies}
Give a concrete example of each of the following network topologies:
  \begin{enumerate}
    \item Bus,
    \item Star,
    \item Fully connected.
  \end{enumerate}

\subsection{TCP connection}
According to TCP (\color{blue}\href{http://tools.ietf.org/html/rfc761}{RFC761 (January 1980)}) \color{black}, what are the sequences used in order to establish a connection between two hosts?

\subsection{TCP or UDP?}
\subsubsection{Sensors}
You are creating a network application using sensors. The sensors can receive requests to change their settings (rate of measurement, range...) and they continuously send their measurements.
  \begin{enumerate}
    \item Should request packets (settings) be sent with UDP or TCP? Why?
    \item Should measurement packets be sent with UDP or TCP? Why?
  \end{enumerate}
\subsubsection{Website}
Does HTTP (\color{blue}\href{http://tools.ietf.org/html/rfc2616}{RFC2616 (June 1999)}) \color{black} rely on TCP or UDP? Why?

\subsection{FTP}
\subsubsection{Is FTP secure?}
According to the file \color{blue}\href{http://teaching.auzias.net/db/ftp-connect.pcap}{ftp-connect.pcap} \color{black} is FTP secure? What could you do to use it more securely?
\subsubsection{FTP and TCP}
According to the file \color{blue}\href{http://teaching.auzias.net/db/ftp-disconnect.pcap}{ftp-disconnect.pcap} \color{black} does FTP respect the TCP protocol to close a connection?

\subsection{DNS}
\subsubsection{Some news}
According to the file \color{blue}\href{http://teaching.auzias.net/db/nslookup.pcap}{nslookup.pcap} \color{black} what is:
  \begin{enumerate}
    \item the DNS server?
    \item the domain name for which the IP address is needed?
    \item the IP address of the domain if any?
  \end{enumerate}n
\subsubsection{Which one?}
According to the file \color{blue}\href{http://teaching.auzias.net/db/nslookup-whoseone.com.pcap}{nslookup-whoseone.com.pcap} \color{black} what is:
  \begin{enumerate}
    \item the DNS server?
    \item the domain name for which the IP address is needed?
    \item the IP address of the domain if any?
  \end{enumerate}

\subsection{Ping-pong}
\subsubsection{Are you there?}
According to the file \color{blue}\href{http://teaching.auzias.net/db/ping.pcap}{ping.pcap} \color{black}:
  \begin{enumerate}
    \item what is the node 127.0.0.1 doing?
    \item Is the node 127.0.0.2 on the network?
  \end{enumerate}
\subsubsection{Who has this IP?}
According to the file \color{blue}\href{http://teaching.auzias.net/db/arp.pcap}{arp.pcap} \color{black} and to ARP (\color{blue}\href{http://tools.ietf.org/html/rfc826}{RFC826 (November 1982)})\color{black}. What is the source trying to do? What is ARP used for? If ever a host does not respond to ping (i.e., for security reasons), how could you check if the host is up anyway ?

\end{document}
